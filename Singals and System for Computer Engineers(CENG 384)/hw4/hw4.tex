\documentclass[10pt,a4paper, margin=1in]{article}
\usepackage{fullpage}
\usepackage{amsfonts, amsmath, pifont}
\usepackage{amsthm}
\usepackage{graphicx}

\usepackage{geometry}
 \geometry{
 a4paper,
 total={210mm,297mm},
 left=10mm,
 right=10mm,
 top=10mm,
 bottom=10mm,
 }
 % Write both of your names here. Fill exxxxxxx with your ceng mail address.
 \author{
  UZUN, Yunus Emre\\
  \texttt{e2172104@ceng.metu.edu.tr}
  \and
  VEFA, Ahmet Dara\\
  \texttt{e2237899@ceng.metu.edu.tr}
}
\title{CENG 384 - Signals and Systems for Computer Engineers \\
Spring 2018-2019 \\
Written Assignment 4}
\begin{document}
\maketitle



\noindent\rule{19cm}{1.2pt}

\begin{enumerate}

\item 
    \begin{enumerate}
    % Write your solutions in the following items.
    \item %write the solution of q1a
    $y[n]=2x[n]+\frac{3}{4}y[n-1]-\frac{y[n-2]}{8} $
    %%e^{-jw}Y(e^{jw})-\frac{1}{8}e^{-2jw}Y(e^{jw}) $
    \item %write the solution of q1b
    $Y(e^{jw})=2X(e^{jw})+\frac{3}{4}e^{-jw}Y(e^{jw})-\frac{1}{8}e^{-2jw}Y(e^{jw}) $\\
    $Y(e^{jw})(1-\frac{3}{4}e^{-jw}+\frac{e^{-2jw}}{8})=2X(e^{jw}) $\\
    $\frac{Y(e^{jw})}{X(e^{jw})}=H(e^{jw})= \frac{1}{(1-\frac{3}{4}e^{-jw}+\frac{e^{-2jw}}{8})}=\frac{-4}{e^{-jw}-2}+\frac{4}{e^{-jw}-4}=2\frac{1}{1-1/2e^{-jw}}-\frac{1}{1-1/4e^{-jw}} $
    \item %write the solution of q1c
    take inverse fourier transform of H \\
    $h[n]=2(\frac{1}{2})^nu[n]-(\frac{1}{4})^nu[n]$
    
    \item %write the solution of q1d
    $X(e^{jw})=\frac{1}{1-1/4e^{-jw}}$\\
    $\frac{Y(e^{jw})}{\frac{1}{1-1/4e^{-jw}}}=\frac{1}{(1-\frac{3}{4}e^{-jw}+\frac{e^{-2jw}}{8})}\xrightarrow{}Y(e^{jw})=(\frac{1}{1-1/4e^{-jw}})(2\frac{1}{1-1/2e^{-jw}}-\frac{1}{1-1/4e^{-jw}})=-4\frac{1}{1-1/4e^{-jw}}+8\frac{1}{1-1/2e^{-jw}}-2(\frac{1}{1-1/4e^{-jw}})^2$\\
    take inverse fourier transform\\
    $y[n]=-4(\frac{1}{4})^nu[n]+8(\frac{1}{2})^nu[n]-2(n+1)(\frac{1}{4})^nu[n]$
    \end{enumerate}


\item %write the solution of q2
$ H(e^{jw})= \frac{-3}{e^{-jw}-3}+2\times(\frac{4}{e^{-jw}-4})=\frac{1}{1-1/3e^{-jw}}-2\times(\frac{1}{1-1/4e^{-jw}}) $\\
take inverse fourier transform of H\\
$h[n]=\frac{1}{3}^nu[n]-2\times(\frac{1}{4})^nu[n]$\\
we know $h_1[n]=\frac{1}{3}^nu[n]$, so $h_2[n]=-2\times(\frac{1}{4})^nu[n]$

\item      
    \begin{enumerate}
    \item %write the solution of q3a
    separate x into 2 parts:\\
    $x_1(t)=\frac{sin2\pi t}{\pi t}$ and $x_2(t)=cos3\pi t $\\
    take their fourier transform:\\
    $X_1(jw)=1$ if $|w|<2\pi$ and 0 otherwise\\
    $X_2(jw)=\pi [\delta(w-3\pi)+\delta(w+3\pi)]$\\
    so fourier transform of x(t) is $X_1(t)+X_2(t)$
    \item %write the solution of q3b
    We found Nyquist Freq. as $3\pi$\\
    $w_s=2w_m\xrightarrow{w_m=3\pi}w_s=6\pi$\\
    $T=\frac{2\pi}{w}\xrightarrow{w_s=6\pi}1/3\xleftarrow{}sampling \ rate$
    \item
    $X_p(jw)=\frac{1}{\frac{1}{3}}\Sigma_{k=-\infty}^{\infty}X(j(w-6\pi k))=3\Sigma_{k=-\infty}^{\infty}X(j(w-6\pi k))$
    \end{enumerate}

\item 
    \begin{enumerate}
    \item %write the solution of q4a
    $N=\frac{2\pi}{w_w}\xrightarrow{w_s=\pi}2$\\
    $X_p(jw)=\frac{1}{t}\Sigma_{\forall k}X(j(w-kw_s))$\\
    $X_d(e^{jw})=X_p(\frac{jw}{T})$\\
    $X_d(e^{jw})=\frac{2w}{\pi}$ if $|w|\leq \frac{\pi}{2}$ 0 otherwise\\
    $X_d(e^{jw})=X_d(e^{j(w+N)})$
    
    \item %write the solution of q4b
    $h[n]=cos\pi n = \frac{1}{2}(e^{j\pi n}+e^{-j\pi n})$, fourier transform of $e^{jw_0 n}$ is $ 2\pi \Sigma_{\forall k}\delta(w-w_0-2\pi k)$, hence\\
    $H(e^{jw})=\frac{1}{2}(2\pi \Sigma_{\forall k }\delta(w-\pi-2\pi k )+2\pi\Sigma_{\forall k }\delta(w+\pi -2\pi k))=\pi( \Sigma_{\forall k }\delta(w-\pi-2\pi k )+\delta(w+\pi -2\pi k))$
    \item %write the solution of q4c
    $Y_d(e^{jw})=\frac{1}{2\pi}X_d(e^{jw})*H(e^{jw})\xrightarrow{convolution\ from\ -\pi \ to\ \pi}\frac{1}{2\pi}\pi(\Sigma_{\forall k}\delta(w-\pi)+\delta(w+\pi)) * X_d(e^{jw})$, just shift $X_d$ to right and left by $\pi$ and we get:\\
    $Y_d(e^{jw})=\frac{w}{\pi}$ if $\frac{\pi}{2}\leq |w|\leq \frac{3\pi}{2}$, 0 otherwise\\
    $Y_d(e^{jw})=X_d(e^{j(+2\pi)})$
    \end{enumerate}


\end{enumerate}
\end{document}

